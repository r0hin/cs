\documentclass[titlepage]{article}
\usepackage{graphicx}
\title{Appendix | HL Computer Science IA}
\author{IBIS Personal Code: jhh041}
\date{Nov 24, 2022}


\begin{document}
\maketitle

\tableofcontents
\pagebreak

\section{A – Interview Transcript}

\begin{center}
  \vspace*{3mm}
  \underline{What is a high-level overview of the problem and desired solution?}
  \vspace*{1mm}
\end{center}
So what I need, there's two things here I want to talk about there is the project itself. But there's also the larger kind of direction of the school. So we are trying to find opportunities to provide for students work together, they're able to engage in a passion project, and they see the interdisciplinary nature of work of projects that work. With this project. In particular, it's a really great idea because it is a bit of a showpiece that we can use to illustrate this. So one of the things that we talk about a lot is that was big data, right? How do we manage data and how do we use them? So we have actually a couple units that actually look at data and data use. And so by by setting up this unit, and being able to demonstrate how you're collecting and sorting data.

I want data collected as a couple of things. One is, I want to have an opportunity where students can see it in action, in a real sense, is connected to a real project to is, you know, we're looking for opportunities for faces across the school to engage in data in an authentic way. And this will provide, I think it's a great second math class or the grade three math class that is looking at different types of energy out there. And so what they can do is they are able to see something that is real, actually look at this and see this created data, they can physically remove this and watch the data change.

\begin{center}
  \vspace*{3mm}
  \underline{What pieces of data will be collected?}
  \vspace*{1mm}
\end{center}

So like GPS, kind of, I think I like to see things I'm actually really open with that. I think it'd be really nice if GPS, time of day, date, obviously, angle of the panel. So and he's really working alongside Nate, who's the person doing the hardware component, if they can, anything that they kind of recommend, but a few pieces of data that specifically would allow us to move location and the time of day. Okay.

\begin{center}
  \vspace*{3mm}
  \underline{For whom will the data be published? Should the data kept accessible?}
  \vspace*{1mm}
\end{center}

Yeah, I would love it to be accessible to everyone within our community, but also one of the things with the design program that we do, because we really try and make our resources, what we can available, people that do not have the same opportunity and access, right? Why would love the site to be one that perhaps has a way to log in as again, to do some more nitty gritty stuff. But if you're, I think of the horizons program, and if there's a teacher from horizons that would like to access the data, they can do that.
Okay, and so with that in mind, I think that the product that you produce, it'd be nice, if you could, they're not going to be here to see it if there was an opportunity to see the different pieces of hardware.

\begin{center}
  \vspace*{3mm}
  \underline{Is data security an important factor?}
  \vspace*{1mm}
\end{center}

It's not super important, because really, we're not, there's not a necessity to collect personal information, I think I'm open to a login system, because that might be nice if there was a login system for them to provide feedback and what we could do better, especially if people outside of our community, right. But in terms of the data itself, there's nothing that is really needed for Super security, I really would prefer something with ease of access. So if you're going to do a login system, you know, finding a way to secure personal information to this data. So I would like anyone to be able to download into the CSV format, I think of the year 10 design coding class, which is a design guide. I would love for next year to have this running in the summer. And then a student downloads that CSV and can do some really cool visualizations with it.


\begin{center}
  \vspace*{3mm}
  \underline{Any other considerations?}
  \vspace*{1mm}
\end{center}

I think one of the things that want you to make sure that you set up as you're in year 12, and you're going to be moving on with this project, and the project is something that I'm hoping, will last beyond you. And I'm hoping that there is structures in place to ensure that it can continue to be developed. So when you develop your codebase, I really, really, really capable programmer, I really want you to think about how you're studying. So really clear, modularization really clear kind of interface in something that is robust. So something when you do pick your dependencies and such begin dependencies that we won't have to necessarily regularly go back and kind of account to so things that are things that are sustainable. So I think of like, you know, I have enough smells like something like Firebase, it's pretty robust. Like, that's not going to change, right, versus doing some secondhand kind of offshoot data collection, I really liked it, it's something that it's we can be secure, that it's going to work. Okay. And keeping everything extensible would be good as well, yeah, in any way possible. Because I would love it if like, I'm hoping what you build will sustain. But again, a student may come in next year and want to add another component. So that said, if there's a way for someone to add in almost like, consider in your code framework, a way to add another piece of data that we haven't considered because I could picture a student come in and say, you know, maybe one day we put some sort of, we automate some process where this rotates, I would like there to be the code to be set up that we can easily interface and that's almost like a success criteria success criteria. The code has an easier almost like a plug and play from a code perspective.

\end{document}
\documentclass[]{article}